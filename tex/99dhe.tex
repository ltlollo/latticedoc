\pagebreak
\section{Esempio di applicazione}
\subsection{ring-LWE-DH key exchange}
Una delle applicazioni dei reticoli, e che fa perciò uso del prodotto
polinomiale in $\mathbb{Z}_q[x]/\langle x^N+1 \rangle$ è lo scambio di chiavi
in crittografia asimmetrica di tipo Diffie-Hellman.\\
\\
Tale scambio avviene seguendo il seguente schema operativo:\\
\begin{tabular} {RL}
    \xymatrix {
        \mathit{Alice} \ar[d]_{(\mathbf{s}_{k}, \mathbf{e})}
        && \mathit{Pub}
        \ar[lld]_{\mathbf{P}_{k}}
        \ar[rrd]^{\mathbf{P}_{k}} \ar@{.}[ddddddd]
        && \mathit{Bob} \ar[d]^{(\mathbf{s}_{k}', \mathbf{e}')}         \\
        \bullet \ar[d]_{\mathbf{P}_{k} \times \mathbf{s}_{k} +
        \mathbf{e} = \mathbf{T}} 
        &&&& \ar[dd]^{\mathbf{P}_{k} \times \mathbf{s}_{k}' +
        \mathbf{e}' = \mathbf{T}'}
        \bullet                                                         \\
        \bullet \ar[rrrrd]^(0.25){\mathbf{T}} \ar[dd]
        &&&&                                                            \\
        &&&& \ar[d]^{(\mathbf{P}_{k}\times \mathbf{s}_{k} +
        \mathbf{e}')
        \times \mathbf{s}_{k}' = \mathbf{V}'}
        \bullet \ar[lllld]_(0.25){\mathbf{T}'}                          \\
        \bullet \ar[dd]_{\mathbf{s}_{k} \times (\mathbf{P}_{k}\times
        \mathbf{s}_{k}' + \mathbf{e}') = \mathbf{V}}
        &&&& \ar[d]^{\mathrm{Round}(\mathbf{V}') = (\mathbf{r},
        \mathbf{s}_{s})}
        \bullet                                                         \\
        &&&& \ar[lllld]_(0.25){\mathbf{r}} \ar[dd]
        \bullet                                                         \\
        \bullet \ar[d]_{\mathrm{Round}(\mathbf{V},\mathbf{r}) =
        \mathbf{s}_s}
        &&&&                                                            \\
        \mathbf{s}_s && \mathit{Eve} && \mathbf{s}_s                    \\
    } &
    \tiny
    \begin{tabular} {|l|}
        \hline
        \\
        $\mathbf{P}_k    : \text{Public key}$            \\
        $\mathbf{s}_k    : \text{Alice's secret key}$    \\
        $\mathbf{s}_k'   : \text{Bob's secret key}$      \\
        $\mathbf{e}      : \text{Alice's error vector}$  \\
        $\mathbf{e}'     : \text{Bob's error vector}$    \\
        $\mathbf{s}_s    : \text{Shared secret}$         \\
        $\mathbf{r}      : \text{Rounding information}$  \\
        $\times : \text{Polynomial product}$    \\
        $+      : \text{Polynomial sum}$        \\
        \\
        \hline
    \end{tabular}
\end{tabular}\\
\\
\\
In un primo momento Alice e Bob, generano due vettori casuali in $\mathbb{Z}_q$,
$(\mathbf{s}_k, \mathbf{s}_k',\mathbf{e},\mathbf{e}')$, distribuiti
gaussianamente attorno a $\frac{q-1}{2}$ con $\sigma = \frac{8}{\sqrt{2\pi}}$,
o in modo uniforme su un intorno $[-b, b]$ di $\frac{q-1}{2}$, con $b$ picclolo
rispetto a $q$. Il primo dei vettori rappresenta la chiave privata ed il
secondo il suo vettore degli errori associato.\\
\pagebreak
\\
Successivamente, data una chiave pubblica $\mathbf{P}_k$, entrambi calcolano il
prodotto della chiave pubblica per la privata più il suo vettore degli errori
($\mathbf{P}_{k} \times \mathbf{s}_{k} + \mathbf{e}$), e lo inviano alla
controparte.\\
\\
Ricevuto il messaggio, entrambi lo moltiplicano per la propria chiave
privata.\\
\\
Bob approssima il vettore così ottenuto mappando i valori in
$[\frac{q-1}{4},\frac{3(q-1)}{4}-1]$ a 1 e i restanti a 0. Invia inoltre ad
Alice un vettore binario $\mathbf{r}$ ottenuto mappando i valori del vettore
originario in $[\frac{q-1}{4},\frac{2(q-1)}{4}-1]\cup[\frac{3(q-1)}{4}+1,q-1]$
a 1, altrimenti a 0.\\
\\
Utilizzando i valori di approssimazione $\mathbf{r}$ reicevuti da Bob, Alice
mappa il vettore ottenuto dal passo precedente da
$[\frac{q-1}{8},\frac{5(q-1)}{8}-1]$ a 1 e 0 altrimenti, se i corrispondenti
valori in $\mathbf{r}$ sono 1, se invece sono 0 da
$[\frac{3(q-1)}{8},\frac{7(q-1)}{8}-1]$ in 1 e 0 altrimenti.\\
\\
Il risultato di entrambe le operazioni di approssimazione è il vettore binario
$\mathbf{s}_s$ di $N$bit, che verrà poi usato come segreto condiviso per
comunicare privatamente attaverso l'uso di schemi di crittografia simmetrica.\\
