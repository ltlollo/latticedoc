\section{Introduzione}
\subsection{Abstract}
Uno dei più rencenti sviluppi della crittografia moderna sono i metodi basati
su reticoli (Lattice), metodi che mostrano promettenti segni di applicabilità
in diversi domini della crittografia, variando da key exchange, signing ed
aprendone di nuovi, quali per esempio quello dell'homomorphic encryption.\\
\\
Tali metodi richiedono lo svolgimento di operazioni di somma e prodotto di
polinomi di grado $N-1$ modulo $q$, dove $N$ potenza del 2, $q$ numero primo in
$\mathbb{Z}$.\\
\\
Essendo tali operazioni molto complesse, ed alla base dell'elaborazione di una
quantità di dati possibilmente molto alta, è necessario che siano implementate
nel modo più efficente possibile.\\
\\
Nel seguente testo si vuole quindi descrivere l'operazione di multiplicazione
polinomiale, facendone una stima di complessità per un'implementazione
diretta, e descrivendone una implementazione che fa uso della NTT,
stimandone la sua complessità, per concludere con una considerazione sulla
implementazione vettorizzata del prodotto di polinomi in
$(N, q) = (1024, \mathtt{0x20008001})$.\\
